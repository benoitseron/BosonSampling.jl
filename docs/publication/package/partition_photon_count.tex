\begin{Shaded}
\begin{Highlighting}[]
\NormalTok{n }\OperatorTok{=} \FloatTok{10}
\NormalTok{m }\OperatorTok{=} \FloatTok{20}

\CommentTok{\# experiment parameters}
\NormalTok{input\_state }\OperatorTok{=} \FunctionTok{first\_modes}\NormalTok{(n,m)}
\NormalTok{interf }\OperatorTok{=} \FunctionTok{RandHaar}\NormalTok{(m)}
\NormalTok{i }\OperatorTok{=} \FunctionTok{Input}\DataTypeTok{\{Bosonic\}}\NormalTok{(input\_state)}

\CommentTok{\# subset selection}
\NormalTok{s }\OperatorTok{=} \FunctionTok{Subset}\NormalTok{(}\FunctionTok{first\_modes}\NormalTok{(}\FunctionTok{Int}\NormalTok{(m}\OperatorTok{/}\FloatTok{2}\NormalTok{),m))}
\NormalTok{part }\OperatorTok{=} \FunctionTok{Partition}\NormalTok{(s)}

\CommentTok{\# want to find all photon counting probabilities}
\NormalTok{o }\OperatorTok{=} \FunctionTok{PartitionCountsAll}\NormalTok{(part)}

\CommentTok{\# define the event and compute probabilities}
\NormalTok{ev }\OperatorTok{=} \FunctionTok{Event}\NormalTok{(i,o,interf)}
\FunctionTok{compute\_probability!}\NormalTok{(ev)}

\CommentTok{\# ... add some output}
\end{Highlighting}
\end{Shaded}
